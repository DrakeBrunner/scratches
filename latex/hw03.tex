\documentclass{article}
\usepackage{enumitem}
\usepackage{amsmath}

\title{ECE345 Assignment 3}
\author{Naoki Mizuno}

\begin{document}
\maketitle

\section*{Part 1}
\subsection*{2-171}
\begin{itemize}
    \item 40\% Highly successful $\rightarrow$ 95\% got good reviews
    \item 35\% Moderately successful $\rightarrow$ 60\% got good reviews
    \item 25\% Poor $\rightarrow$ 10\% got good reviews
\end{itemize}

\begin{enumerate}[label=(\alph*)]
    % a
    \item
        Let $H$, $M$, $P$ denote the event that a product is highly
        successful, moderately successful, and poor, respectively. Also, let
        $G$ denote the event that the product got good reviews.

        % P(G and H)
        \begin{equation*}
            \begin{aligned}
                P(G \cap H) &= P(G \vert H) P(H)\\
                           &= 0.95 \times 0.4 \\
                           &= 0.38
            \end{aligned}
        \end{equation*}
        % P(G and M)
        \begin{equation*}
            \begin{aligned}
                P(G \cap M) &= P(G \vert M) P(M)\\
                           &= 0.6 \times 0.35 \\
                           &= 0.21
            \end{aligned}
        \end{equation*}
        % P(G and P)
        \begin{equation*}
            \begin{aligned}
                P(G \cap P) &= P(G \vert P) P(P)\\
                            &= 0.1 \times 0.25 \\
                            &= 0.025
            \end{aligned}
        \end{equation*}

        Since $P(G \cap H)$, $P(G \cap M)$, $P(G \cap P)$ are mutually
        exclusive,

        % Calculate P(G)
        \begin{equation*}
            \begin{aligned}
                P(G) &= P(G \cap H) + P(G \cap M) + P(G \cap P) \\
                     &= 0.38 + 0.19 + 0.025 \\
                     &= 0.595
            \end{aligned}
        \end{equation*}

    % b
    \item
        \begin{equation*}
            \begin{aligned}
                P(H \vert G) &= \frac{P(H \cap G)}{P(G)} \\
                &= \frac{0.38}{0.595} \\
                &= 0.638655462
            \end{aligned}
        \end{equation*}

    % c
    \item
        \begin{equation*}
            P(H \vert G') = \frac{P(H \cap G')}{P(G')}
        \end{equation*}
        \begin{equation*}
            \begin{aligned}
                P(H \cap G') &= 0.4 \times 0.05 \\
                             &= 0.02
            \end{aligned}
        \end{equation*}
        \begin{equation*}
            \begin{aligned}
                P(G') &= 1 - P(G) \\
                      &= 0.405
            \end{aligned}
        \end{equation*}

        Thus,

        \begin{equation*}
            \begin{aligned}
                P(H \vert G') &= \frac{0.02}{0.405} \\
                              &= 0.049382716
            \end{aligned}
        \end{equation*}
\end{enumerate}

\subsection*{2-179}
Let $S$ denote the event that a message is spam. Let $F$ denote the event that
a message contains the word ``free''.

\begin{enumerate}[label=(\alph*)]
    % a
    \item
        \begin{equation*}
            P(F) = P(S \cap F) + P(S' \cap F) \\
        \end{equation*}
        \begin{equation*}
            \begin{aligned}
                P(S \cap F) &= P(F \vert S) P(S) \\
                            &= 0.6 \times 0.2 \\
                            &= 0.12
            \end{aligned}
        \end{equation*}
        \begin{equation*}
            \begin{aligned}
                P(S' \cap F) &= P(F \vert S') P(S') \\
                             &= 0.04 \times 0.8 \\
                             &= 0.032
            \end{aligned}
        \end{equation*}

        Thus,

        \begin{equation*}
            \begin{aligned}
                P(F) &= 0.12 + 0.032 \\
                     &= 0.152
            \end{aligned}
        \end{equation*}

    % b
    \item
        \begin{equation*}
            \begin{aligned}
                P(S \vert F) &= \frac{P(S \cap F)}{P(F)} \\
                             &= \frac{0.12}{0.152} \\
                             &= 0.789473684
            \end{aligned}
        \end{equation*}

    % c
    \item
        \begin{equation*}
            \begin{aligned}
                P(S' \vert F') &= \frac{P(S' \cap F')}{P(F')} \\
                               &= \frac{(P(S \cup F))'}{P(F')} \\
                               &= \frac{(P(S) + P(F) - P(S \cap F))'}{P(F')} \\
                               &= \frac{1 - (0.2 + 0.152 - 0.12)}{1 - 0.152} \\
                               &= 0.905660377
            \end{aligned}
        \end{equation*}
\end{enumerate}

\subsection*{2-183}
\begin{enumerate}[label=(\alph*)]
    \item Discrete because samples will be counted in integers.
    \item Continuous because weight is not discrete.
    \item Discrete because number of molecules can be counted in integers.
    \item Continuous because concentration is not discrete.
    \item Continuous because current is not discrete.
\end{enumerate}

\section*{Part 2}
\subsection*{3-17}
\begin{equation*}
    X = \{-2, -1, 0, 1, 2\}
\end{equation*}
\begin{equation*}
    \begin{aligned}
        \sum_{i = 1}^{n} f(x_i) &= f(-2) + f(-1) + f(0) + f(1) + f(2) \\
                                &= 0.2 + 0.4 + 0.1 + 0.2 + 0.1 \\
                                &= 1
    \end{aligned}
\end{equation*}
\begin{equation*}
    f(x_i) = P(X = x_i) \geq 0
\end{equation*}

\begin{enumerate}[label=(\alph*)]
    % a
    \item
        Since all X values are less than or equal to 2,
        \begin{equation*}
            P(X \leq 2) = 1
        \end{equation*}

    % b
    \item
        \begin{equation*}
            \begin{aligned}
                P(X > -2) &= P(X = -1) + P(X = 0) + ... + P(X = 2) \\
                            &= 1 - P(X = -2) \\
                            &= 0.8
            \end{aligned}
        \end{equation*}

    % c
    \item
        \begin{equation*}
            \begin{aligned}
                P(-1 \leq X \leq 1) &= P(X = -1) + P(X = 0) + P(X = 1) \\
                                    &= 0.7
            \end{aligned}
        \end{equation*}

    % d
    \item
        \begin{equation*}
            \begin{aligned}
                P(X \leq -1 \text{ or } X = 2) &= P(X = -1) + P(X = -2) + P(X = 2) \\
                                      &= 0.7
            \end{aligned}
        \end{equation*}
\end{enumerate}

\subsection*{3-19}
\begin{equation*}
    f(x) = \frac{2x + 1}{25} \text{ where } x = \{0, 1, 2, 3, 4\}
\end{equation*}

\begin{equation*}
    \begin{aligned}
        \sum_{i = 1}^{n} f(x_i) &= \frac{1}{25} + \frac{3}{25} + \frac{5}{25}
                                    + \frac{7}{25} + \frac{9}{25} \\
                                &= \frac{25}{25} \\
                                &= 1
    \end{aligned}
\end{equation*}

Also,

\begin{equation*}
    f(x_i) = P(X = i) \geq 0
\end{equation*}

\begin{enumerate}[label=(\alph*)]
    % a
    \item
        \begin{equation*}
            \begin{aligned}
                P(X = 4) &= \frac{2 \times 4 + 1}{25} \\
                         &= \frac{9}{25}
            \end{aligned}
        \end{equation*}
    % b
    \item
        \begin{equation*}
            \begin{aligned}
                P(X \leq 1) &= P(X = 0) + P(X = 1) \\
                            &= \frac{1}{25} + \frac{3}{25} \\
                            &= \frac{4}{25}
            \end{aligned}
        \end{equation*}

    % c
    \item
        \begin{equation*}
            \begin{aligned}
                P(2 \leq X < 4) &= P(X = 2) + P(X = 3) \\
                                &= \frac{2 \times 2 + 1}{25}
                                    + \frac{2 \times 3 + 1}{25} \\
                                &= \frac{12}{25}
            \end{aligned}
        \end{equation*}

    % d
    \item
        \begin{equation*}
            \begin{aligned}
                P(X > -10) &= P(X = 0) + P(X = 1) + ... + P(X = 4) \\
                           &= 1
            \end{aligned}
        \end{equation*}
\end{enumerate}

\subsection*{3-25}
Probability Mass Function is:

\begin{equation*}
    \begin{aligned}
        f(x) = 0.2^{3-x} \cdot 0.8^x \text{ for } X = \{0, 1, 2, 3\}
    \end{aligned}
\end{equation*}

\subsection*{3-39}

\begin{equation*}
    F(x) = 
    \left\{
        \begin{array}{rr@{\hspace{5pt}}l}
            0   &         & x < -2 \\
            0.2 & -2 \leq & x < -1 \\
            0.6 & -1 \leq & x < 1 \\
            0.7 & 0  \leq & x < 1 \\
            0.9 & 1  \leq & x < 2 \\
            1   & 2  \leq & x
        \end{array}
        \right.
\end{equation*}

\begin{enumerate}[label=(\alph*)]
    \item $P(X \leq 1.25) = 0.7$
    \item $P(X \leq 2.2) = 1$
    \item $P(-1.1 < X \leq 1) = 0.6$
    \item $P(X > 0) = 0.6$
\end{enumerate}

\subsection*{3-51}
\begin{enumerate}[label=(\alph*)]
    \item
        $P(X \leq 50) = 1$
    \item
        $P(X \leq 40) = 0.75$
    \item
        $P(40 \leq X \leq 60) = 0.25$
    \item
        $P(X < 0) = 0.25$
    \item
        $P(0 \leq X < 10) = 0$
    \item
        $P(-10 < X < 10) = 0$
\end{enumerate}

\subsection*{3-59}
\begin{description}
    \item[Mean]
        $0$
    \item[Variance]
        \begin{equation*}
            \begin{aligned}
                \mu &= (-2 \times 0.2) + (-1 \times 0.4) + (0 \times 0.1) + (1 \times 0.2) + (2 \times 0.1) \\
                    &= -0.4 \\
            \end{aligned}
        \end{equation*}
        \begin{equation*}
            \sum_{i = 1}^{5} (x_i + 0.4)^2 f(x_i) = 1.64
        \end{equation*}
\end{description}
\subsection*{3-75}
The result can be either ``0 device fails'', ``1 device fails'', or ``2
devices fail''.

\begin{table}[h]
    \centering
    \begin{tabular}{|c|c|c|c|}
        \hline
        $X$ & 0 & 1 & 2 \\
        \hline
        $f(x)$ & 0.72 & 0.26 & 0.02 \\
        \hline
    \end{tabular}
\end{table}

The mean (i.e. the expectation) can be calculated by:
\[
    0 \times 0.72 + 1 \times 0.26 + 2 \times 0.02 = 0.3
\]

\subsection*{3-84}
Since
\[
    \mu = \sum_{x} xf(x)
\]
, multiplying each values (i.e. $f(x_i)$) in $X$ by a constant $c$ gives

\[
    \mu' = \sum_{x} xcf(x)
\]

Since
\[
    \sum_{x} cf(x) = c
\]
,
\[
    \mu' = c\mu
\]

Also, since variance is calculated by
\begin{equation*}
    \begin{aligned}
        \sum_{x} x^2f(x) - \mu'^2 \\
        = c^2 \sum_{x} x^2f(x) - \mu^2
    \end{aligned}
\end{equation*}

\end{document}
